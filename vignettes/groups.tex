% Created 2014-09-26 Fri 10:38
\documentclass[11pt]{article}
\usepackage[utf8]{inputenc}
\usepackage[T1]{fontenc}
\usepackage{fixltx2e}
\usepackage{graphicx}
\usepackage{longtable}
\usepackage{float}
\usepackage{wrapfig}
\usepackage{rotating}
\usepackage[normalem]{ulem}
\usepackage{amsmath}
\usepackage{textcomp}
\usepackage{marvosym}
\usepackage{wasysym}
\usepackage{amssymb}
\usepackage{hyperref}
\tolerance=1000
\usepackage{fullpage}
\author{Chris Wallace}
\date{2014-04-25 Fri}
\title{Groups, tags, and snp.picker}
\hypersetup{
  pdfkeywords={},
  pdfsubject={},
  pdfcreator={Emacs 24.3.1 (Org mode 8.2.7c)}}
\usepackage{Sweave}
\begin{document}

\maketitle
\tableofcontents

%\VignetteIndexEntry{Groups, tags, and snp.picker}

\section{Grouping SNPs}
\label{sec-1}

We often (usually) cannot resolve an association to a single causal
variant, and one of the aims of GUESSFM is to generate groups of SNPs,
one of which is likely to be causal.

\section{The groups and tags classes}
\label{sec-2}
The \texttt{groups} and \texttt{tags} classes are two ways to represent the same
information, but use different structures which might be more
convenient for one function or another.

\begin{Schunk}
\begin{Sinput}
> library(GUESSFM)
> showClass("groups")
\end{Sinput}
\begin{Soutput}
Class "groups" [package "GUESSFM"]

Slots:
                          
Name:      .Data      tags
Class:      list character

Extends: 
Class "list", from data part
Class "vector", by class "list", distance 2
Class "AssayData", by class "list", distance 2
Class "vectorORfactor", by class "list", distance 3
\end{Soutput}
\begin{Sinput}
> showClass("tags")
\end{Sinput}
\begin{Soutput}
Class "tags" [package "GUESSFM"]

Slots:
                          
Name:      .Data      tags
Class: character character

Extends: 
Class "character", from data part
Class "vector", by class "character", distance 2
Class "data.frameRowLabels", by class "character", distance 2
Class "SuperClassMethod", by class "character", distance 2
Class "atomicVector", by class "character", distance 2
Class "index", by class "character", distance 2
Class "characterORconnection", by class "character", distance 2
Class "atomic", by class "character", distance 2
Class "characterORNULL", by class "character", distance 2
Class "EnumerationValue", by class "character", distance 2
Class "characterORMIAME", by class "character", distance 2
Class "characterOrRle", by class "character", distance 2
Class "vectorORfactor", by class "character", distance 3
\end{Soutput}
\end{Schunk}

\section{Automatic group building based on marginal posterior probabilities}
\label{sec-3}
Once models have been fitted, we can average over them to find
marginal posterior probabilities for each SNP.  We make the assumption
that SNPs in a group should display the following properties:
\begin{itemize}
\item the sum of MPI should be no more than 1
\item SNPs in a group should not occur in the same model together very
often
\item SNPs in a group should be correlated (r$^{\text{2}}$>0.5), and the MPI should tail
off from the primary SNP with r$^{\text{2}}$
\end{itemize}

\texttt{snp.picker()} aims to create such groups in an automated manner,
given the summary of a GUESS run and a genotype dataset from which r$^{\text{2}}$
may be calculated.  It returns an object of class \texttt{snppicker} which is
an extension of the \texttt{groups} class and contains information to allow
plotting of the relationship between r$^{\text{2}}$ and MPI that was used to
define a group.

\begin{Schunk}
\begin{Sinput}
> showClass("snppicker")
\end{Sinput}
\begin{Soutput}
Class "snppicker" [package "GUESSFM"]

Slots:
                          
Name:  plotsdata    groups
Class:      list      list
\end{Soutput}
\end{Schunk}

\section{Manual corrections}
\label{sec-4}
Plotting an object of class \texttt{snppicker} should be informative.  The
algorithm underlying \texttt{snp.picker()} has been designed so it is more
likely to err by splitting a single group in two than by joining what
might be two separate groups.  If you suspect two groups should be
merged, you can check the assumptions above using \texttt{check.merge()}, and
perform a merge using \texttt{group.merge()}.
% Emacs 24.3.1 (Org mode 8.2.7c)
\end{document}
