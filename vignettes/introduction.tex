% Created 2014-09-26 Fri 10:39
\documentclass[11pt]{article}
\usepackage[utf8]{inputenc}
\usepackage[T1]{fontenc}
\usepackage{fixltx2e}
\usepackage{graphicx}
\usepackage{longtable}
\usepackage{float}
\usepackage{wrapfig}
\usepackage{rotating}
\usepackage[normalem]{ulem}
\usepackage{amsmath}
\usepackage{textcomp}
\usepackage{marvosym}
\usepackage{wasysym}
\usepackage{amssymb}
\usepackage{hyperref}
\tolerance=1000
\usepackage{fullpage}
\author{Chris Wallace}
\date{2013-12-03 Tue}
\title{GUESSFM: using GUESS for Fine Mapping}
\hypersetup{
  pdfkeywords={},
  pdfsubject={},
  pdfcreator={Emacs 24.3.1 (Org mode 8.2.7c)}}
\usepackage{Sweave}
\begin{document}

\maketitle
\tableofcontents

%\VignetteIndexEntry{GUESSFM Introduction}

\section{Introduction}
\label{sec-1}

\href{http://www.bgx.org.uk/software/guess.html}{GUESS} is software for Bayesian variable selection and model averaging
using a stochastic search algorithm to visit the most likely models
via MCMC.  There also exists an R package, \href{http://cran.r-project.org/web/packages/R2GUESS/index.html}{R2GUESS}, which is a wrapper
for running GUESS, and provides useful diagnostic and summary
functions.

This package, GUESSFM, aims to extend GUESS for fine mapping, defined
here as the task of identifying the most likely set of causal
variants.  The are particular challenges for fine mapping, mostly
related to the density of SNP markers.  Nonetheless, this vignette
will at first focus on an example dataset that comes with R2GUESS.

\section{R2GUESS Strategy}
\label{sec-2}

Although GUESS uses g-priors, which inhibits visiting models with SNPs
in high LD, we have found that running GUESS with SNPs in complete LD
can lead to instability.  R2GUESS was created purely to allow running
GUESS on a tagged set of SNPs to approximate the posterior model
space, and then expand the most interesting models to include their tags.  It is perhaps useful to give an example of expanding a model.

Suppose we begin tagging, and determine a couple of \emph{tag groups} as follows:

\begin{center}
\begin{tabular}{ll}
Tag & SNPs in tag group\\
$A$ & $A_1$, $A_2$\\
$B$ & $B_1$\\
\end{tabular}
\end{center}

R2GUESS will remove SNPs A1, A2 and B1, and will save a file
containing an object of class \texttt{tags} containting the information in
the above table.  Then, suppose after GUESS has run on the set of tag
SNPs, model $(A,B)$ is one of the most interesting.  GUESS FM will
`expand' this model to the set of models ${(A,B), (A_1,B), (A_2,B),
(A,B_1), (A_1,B_1), (A_2,B_1)}$.  

We assess each model using Bayes Factors.  The Bayes Factor for model
$M_i$ is 

$$BF_{i0} = \frac{P(M_i | \text{data} )}{P(M_0 | \text{data})}.$$

In a fully Bayesian analysis, the individual model
probabilities are calculated by integrating over prior distributions
for the model parameters (regression coefficients, for example).  This
is what GUESS does.  However, it assumes a linear model when often we
will want to fine map a disease trait and therefore use a logistic
model.  Further, we cannot go back and run the expanded models through
GUESS.  Instead, we use Approximate Bayes Factors, based on the BIC:

$$-2 \log{ABF_{i0}} = BIC(M_i) - BIC(M_0).$$

These are calculated within R via the \texttt{glm} and \texttt{BIC} functions.  

At what threshold should you tag?  Probably, as high as you can go
without inducing instability in GUESS.  We have found $r^2=0.99$ to
work in some large datasets, but your mileage may vary.  In the
examples below we use a lower threshold purely for demonstration.

\section{Simulate some data}
\label{sec-3}

We start with using some sample data from the snpStats package
including 20 SNPs, and simulating a quantitative trait that depends
on 3 causal SNPs.

\begin{Schunk}
\begin{Sinput}
> library(snpStats)
> data(for.exercise, package="snpStats")
> set.seed(12346)
> X <- snps.10[,101:120]
> n <- nrow(X)
> causal <- c("rs1555897","rs7069505")
> Y <- rnorm(n,mean=as.numeric(X[,causal[1]]))*sqrt(0.2) +
+   rnorm(n,mean=as.numeric(X[,causal[2]]))*sqrt(0.2) +
+   rnorm(n)*sqrt(0.6)
\end{Sinput}
\end{Schunk}

\texttt{X} contains some missing genotypes, but no SNPs with such a low call
rate we would worry in a large study. Still, the rest of the analysis
is easier to interpret for the purposes of a vignette if we fill in
the missing values.

\begin{Schunk}
\begin{Sinput}
> library(GUESSFM)
> summary(col.summary(X))
\end{Sinput}
\begin{Soutput}
     Calls         Call.rate      Certain.calls      RAF         
 Min.   :984.0   Min.   :0.9840   Min.   :1     Min.   :0.04651  
 1st Qu.:988.5   1st Qu.:0.9885   1st Qu.:1     1st Qu.:0.16658  
 Median :989.0   Median :0.9890   Median :1     Median :0.41617  
 Mean   :989.7   Mean   :0.9897   Mean   :1     Mean   :0.41482  
 3rd Qu.:990.0   3rd Qu.:0.9900   3rd Qu.:1     3rd Qu.:0.62680  
 Max.   :998.0   Max.   :0.9980   Max.   :1     Max.   :0.84747  
      MAF               P.AA              P.AB              P.BB         
 Min.   :0.04651   Min.   :0.01921   Min.   :0.08898   Min.   :0.002022  
 1st Qu.:0.14523   1st Qu.:0.14170   1st Qu.:0.22626   1st Qu.:0.029575  
 Median :0.31430   Median :0.33984   Median :0.42095   Median :0.172658  
 Mean   :0.27944   Mean   :0.40774   Mean   :0.35487   Mean   :0.237383  
 3rd Qu.:0.37406   3rd Qu.:0.69520   3rd Qu.:0.47258   3rd Qu.:0.392982  
 Max.   :0.49298   Max.   :0.90900   Max.   :0.50101   Max.   :0.732323  
     z.HWE        
 Min.   :-3.5140  
 1st Qu.:-1.1135  
 Median : 0.1907  
 Mean   :-0.4489  
 3rd Qu.: 0.4224  
 Max.   : 1.1354  
\end{Soutput}
\begin{Sinput}
> X <- impute.missing(X)
\end{Sinput}
\begin{Soutput}
20 to impute
1 .SNPs tagged by multiple tag haplotypes (saturated model): 1
2 .SNPs tagged by multiple tag haplotypes (saturated model): 1
3 .SNPs tagged by multiple tag haplotypes (saturated model): 1
4 .SNPs tagged by a single SNP: 1
5 .SNPs tagged by a single SNP: 1
6 .SNPs tagged by a single SNP: 1
7 .SNPs tagged by a single SNP: 1
8 .SNPs tagged by multiple tag haplotypes (saturated model): 1
9 .SNPs tagged by multiple tag haplotypes (saturated model): 1
10 .SNPs tagged by multiple tag haplotypes (saturated model): 1
11 .SNPs tagged by multiple tag haplotypes (saturated model): 1
12 .SNPs tagged by multiple tag haplotypes (saturated model): 1
13 .SNPs tagged by multiple tag haplotypes (saturated model): 1
14 .SNPs tagged by multiple tag haplotypes (saturated model): 1
15 .SNPs tagged by multiple tag haplotypes (saturated model): 1
16 .SNPs tagged by multiple tag haplotypes (saturated model): 1
17 .SNPs tagged by multiple tag haplotypes (saturated model): 1
18 .SNPs tagged by multiple tag haplotypes (saturated model): 1
19 .SNPs tagged by multiple tag haplotypes (saturated model): 1
20 .SNPs tagged by multiple tag haplotypes (saturated model): 1

coercing object of mode  numeric  to SnpMatrix
\end{Soutput}
\begin{Sinput}
> summary(col.summary(X))
\end{Sinput}
\begin{Soutput}
     Calls        Call.rate Certain.calls      RAF              MAF        
 Min.   :1000   Min.   :1   Min.   :1     Min.   :0.0460   Min.   :0.0460  
 1st Qu.:1000   1st Qu.:1   1st Qu.:1     1st Qu.:0.1660   1st Qu.:0.1457  
 Median :1000   Median :1   Median :1     Median :0.4158   Median :0.3145  
 Mean   :1000   Mean   :1   Mean   :1     Mean   :0.4145   Mean   :0.2794  
 3rd Qu.:1000   3rd Qu.:1   3rd Qu.:1     3rd Qu.:0.6269   3rd Qu.:0.3739  
 Max.   :1000   Max.   :1   Max.   :1     Max.   :0.8465   Max.   :0.4940  
      P.AA             P.AB             P.BB             z.HWE        
 Min.   :0.0200   Min.   :0.0880   Min.   :0.00200   Min.   :-3.5284  
 1st Qu.:0.1410   1st Qu.:0.2268   1st Qu.:0.02925   1st Qu.:-1.0185  
 Median :0.3400   Median :0.4220   Median :0.17150   Median : 0.2031  
 Mean   :0.4079   Mean   :0.3551   Mean   :0.23700   Mean   :-0.4348  
 3rd Qu.:0.6957   3rd Qu.:0.4733   3rd Qu.:0.39275   3rd Qu.: 0.3971  
 Max.   :0.9100   Max.   :0.5020   Max.   :0.73100   Max.   : 1.1292  
\end{Soutput}
\end{Schunk}

Looking at the LD, we see this is a region in which D' (above the
diagonal) is very high, whilst $r^2$ can be high between some SNPs,
and with moderately strong $r^2 \simeq 0.7$ between two of our causal
SNPs:
\begin{Schunk}
\begin{Sinput}
> ld <- show.ld(X=X)
\end{Sinput}
\end{Schunk}


\section{Running GUESS and reading output}
\label{sec-4}

First, via GUESSFM's wrapper, tagging at $r^2=0.8$.  This is
considerably lower than suggested above, and is used here purely for
demonstration as there is limited strong LD in this dataset.

\begin{verbatim}
## THIS DIRECTORY WILL HOLD ALL THE GUESS INPUT AND OUTPUT 
## AND WILL BE CREATED IF IT DOESN'T ALREADY EXIST
mydir <- tempfile() 
run.bvs(X,Y,gdir=mydir,backend="guess",tag.r2=0.8) ## THIS CAN TAKE A WHILE...
## what files were created?
list.files(mydir)

## read output with GUESSFM
d <- read.snpmod(mydir)

## examine the best models and SNPs with greatest marginal support within the tagged data.
best.models(d)
best.snps(d)
\end{verbatim}


Note that both \texttt{best.models} and \texttt{best.snps} allow you to specify thresholds for
how to determine "best".  See their help pages for details.

The tags created within the \texttt{run.bvs} function are saved to a
\texttt{tags.RData} file under \texttt{mydir} and can be examined.

\begin{Schunk}
\begin{Sinput}
> (load(file.path(mydir,"tags.RData")))